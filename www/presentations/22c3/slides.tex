\documentclass[compress]{beamer}

\usepackage{german}
\usepackage{url}

\title{Writing better code (in Dylan)}
\subtitle{Fast development of object-oriented functional programs}

\author{Andreas Bogk and Hannes Mehnert}

\begin{document}
\date{Chaos Communication Congress, 27.12.2005}
\frame{\titlepage}

\begin{frame}
  \frametitle{Requirements for programming languages}
  \begin{itemize}
  \item universal
  \item powerful
  \item easy to learn
  \item performant
  \item easy to read
  \item open source implementation
  \item tools (debugger, profiler, IDE)
  \item libraries, frameworks
  \item secure!
  \end{itemize}
\end{frame}

\begin{frame}
  \frametitle{Dylan competes!}
  ICFP Programming Contest
  \begin{itemize}
  \item 2001: Second Prize
  \item 2003: Judges Prize
  \item 2005: Second Prize and Judges Prize
  \end{itemize}
In 2005, we competed against 168 teams!
\end{frame}

\begin{frame}
  \frametitle{History of Dylan}
  \begin{itemize}
  \item dialect of lisp
  \item Ralph, the programming language for Apple Newton
  \item Apple, CMU, Harlequin
  \item Dylan Interim Reference Manual
  \item Dylan Reference Manual (DRM)
  \end{itemize}
  since DRM no longer prefix (lisp) syntax
\end{frame}

\begin{frame}
  \frametitle{Dylan Features}
  \begin{itemize}
  \item Object-Oriented (from the grounds up)
  \item Multiple Inheritance (with superclass linearization)
  \item Multiple Dispatch
  \item First class functions
  \item First class types
  \item Dynamic typing
  \item Metaprogramming
  \item Optimistic Type Inferencing
  \end{itemize}
\end{frame}

\begin{frame}
  \frametitle{Apple Dylan}
  \begin{itemize}
  \item technology release (based on MCL)
  \item Apple Cambridge labs
  \item implementation on 68k, later PowerPC
  \item 1996 abandoned for lack of money
  \end{itemize}
\end{frame}

\begin{frame}
  \frametitle{CMU}
  \begin{itemize}
  \item gwydion project
  \item goal: development environment
  \item DARPA funded between 1994 and 1998
  \item dylan interpreter in C
  \item dylan compiler to C written in dylan
  \item since 1994 open source license (mostly BSD)
  \item since 1998 open development process
  \end{itemize}
\end{frame}

\begin{frame}
  \frametitle{Harlequin}
  \begin{itemize}
  \item dylan compiler written in dylan
  \item developers had many experience (lisp machine, LispWorks)
  \item native compiler with IDE (only on win32 so far)
  \item debugger, profiler, interactor, hot code update
  \item command line compiler for linux/x86
  \item originally a commercial development, since 2004 open source license (LGPL)
  \item large scale professional development (30 person years work just for the garbage collector)
  \end{itemize}
\end{frame}

\begin{frame}
  \frametitle{Existing libraries}
  \begin{itemize}
  \item DUIM (Dylan User Interface Manager)
  \item corba (2.0, some 2.2 features)
  \item ODBC
  \item network
  \item regular expressions
  \item dood (persistent object store)
  \item file system
  \item XML parser
  \item C interfaces to png, pdf, postgresql, sdl, opengl
  \item stand alone web server and proof of concept wiki
  \item ...
  \end{itemize}
\end{frame}

%\begin{frame}
%  \begin{itemize}
%  \item Einordnung in Programmiersprachen
%  \end{itemize}
%\end{frame}

\begin{frame}[fragile]
  \frametitle{Syntax}
Algol like syntax:
  \begin{verbatim}
begin
  for (i from 0 below 9)
    format-out("Hello world");
  end for;
end
  \end{verbatim}
\end{frame}

\begin{frame}
  \frametitle{Naming Conventions}
  \begin{itemize}
  \item allowed in names: +=-*$<>$
  \item - instead of \_
  \item classes begin and end with angle brackets: $<$number$>$
  \item global variables begin and end with asterisks: *machine-state*
  \item program constants begin with a dollar sign: \$pi
  \item predicate functions end with a question mark: even?
  \item destructive functions end with exclamation mark: reverse!
  \item getters and setters: element element-setter
  \end{itemize}
\end{frame} 

\begin{frame}
  \frametitle{Dynamically and strongly typed}
  \begin{itemize}
  \item strong vs weak typing
  \item static vs dynamic typing
  \end{itemize}
\end{frame}

\begin{frame}
  \frametitle{Object oriented}
  \begin{itemize}
  \item class based object system
  \item everything is inherited from class $<$object$>$
  \item multiple inheritance, but the right way: superclass linearization
  \item difference to widely deployed object oriented programming languages: functions are not part of classes
  \end{itemize}
\end{frame}

\begin{frame}[fragile]
  \frametitle{Class definition}
  \begin{verbatim}
define class <square> (<rectangle>)
  slot x :: <number> = 0, init-keyword: x:;
  slot y :: <number> = 0, init-keyword: y:;
  constant slot width :: <number>,
     required-init-keyword: width:;
end class;  
  \end{verbatim}
\end{frame}

%\begin{frame}
%  \begin{itemize}
%  \item Garbage collection
%  \end{itemize}
%\end{frame}

\begin{frame}[fragile]
  \frametitle{Keyword arguments}
  \begin{verbatim}
define function describe-list
    (my-list :: <list>, #key verbose?) => ()
  format(*standard-output*,
         "{a <list>, size: %d",
         my-list.size);
  if (verbose?)
    format(*standard-output*, ", elements:");
    for (item in my-list)
      format(*standard-output*, " %=", item);
    end for;
  end if;
  format(*standard-output*, "}");
end function;
  \end{verbatim}
\end{frame}

\begin{frame}
  \frametitle{Higher order functions}
  \begin{itemize}
  \item anonymous functions (lambda calculus)
  \item closures
  \item curry, reduce, map, do
  \item function composition
  \end{itemize}
\end{frame}

\begin{frame}[fragile]
  \frametitle{Anonymous functions and closures}
  \begin{verbatim}
define function make-linear-mapper
    (times :: <integer>, plus :: <integer>)
 => (mapper :: <function>)
  method (x)
    times * x + plus;
  end method;
end function;

define constant times-two-plus-one =
  make-linear-mapper(2, 1);

times-two-plus-one(5);
// Returns 11.
  \end{verbatim}
\end{frame}

\begin{frame}[fragile]
  \frametitle{Curry, reduce, map}
  \begin{verbatim}
let printout = curry(print-object, *standard-output*);
do(printout, #(1, 2, 3));

reduce(\+, 0, #(1, 2, 3)) // returns 6

reduce1(\+, #(1, 2, 3)) //returns 6

map(\+, #(1, 2, 3), #(4, 5, 6))
 //returns #(5, 7, 9)
  \end{verbatim}
\end{frame}

\begin{frame}[fragile]
  \frametitle{Function composition, interfacing to C}
  \begin{verbatim}
define interface
  #include "ctype.h",
    import: {"isalpha" => is-alphabetic?,
	     "isdigit" => is-numeric?},
    map: {"int" => <boolean>};
end interface;

define constant is-alphanumeric? =
  disjoin(is-alphabetic?, is-numeric?);
  \end{verbatim}
\end{frame}

\begin{frame}[fragile]
  \frametitle{Generic functions}
  \begin{verbatim}
define method double
    (s :: <string>) => result
  concatenate(s, s);
end method;

define method double
    (x :: <number>) => result
  2 * x;
end method;
  \end{verbatim}
\end{frame}

\begin{frame}[fragile]
  \frametitle{Multiple dispatch}
  \begin{footnotesize}
  \begin{verbatim}
define method inspect-vehicle
  (vehicle :: <vehicle>, i :: <inspector>) => ();
  look-for-rust(vehicle);
end;
define method inspect-vehicle
    (car :: <car>, i :: <inspector>) => ();
  next-method(); // perform vehicle inspection
  check-seat-belts(car);
end;
define method inspect-vehicle
    (truck :: <truck>, i :: <inspector>) => ();
  next-method(); // perform vehicle inspection
  check-cargo-attachments(truck);
end;
define method inspect-vehicle
    (car :: <car>, i :: <state-inspector>) => ();
  next-method(); // perform car inspection
  check-insurance(car);
end;
  \end{verbatim}
  \end{footnotesize}
\end{frame}

\begin{frame}[fragile]
  \frametitle{Optional type restrictions of bindings}
  \begin{verbatim}
define method foo
    (a :: <number>, b :: <number>)
  let c = a + b;
  let d :: <integer> = a * b;
  c := "foo";
  d := "bar"; // Type error!
end
  \end{verbatim}
  Serves on the one hand as assert, on the other hand type inference.
\end{frame}

\begin{frame}[fragile]
  \frametitle{Macros}
  \begin{verbatim}
define macro with-open-file
  { with-open-file (?stream:variable = ?locator:expression,
                    #rest ?keys:expression)
      ?body:body
    end }
  => { begin
         let ?stream = #f;
         block ()
           ?stream := open-file-stream(?locator, ?keys);
           ?body
         cleanup
           if (?stream & stream-open?(?stream))
             close(?stream)
           end;
         end
       end }
end macro with-open-file;
  \end{verbatim}
\end{frame}


%warum das performt

\begin{frame}[fragile]
  \frametitle{A simple for-loop...}
  \begin{verbatim}
let collection = #[1, 2, 3];
for (i in collection)
  format-out("%=\n", i);
end for;
  \end{verbatim}
\end{frame}

\begin{frame}[fragile]
  \frametitle{... is in real a macro with iterator ...}
  \begin{verbatim}
let (initial-state, limit, next-state, finished-state?,
     current-key, current-element) =
  forward-iteration-protocol(collection);
local method repeat (state)
   block (return)
     unless (finished-state?(collection, state, limit))
       let i = current-element(collection, state);
       format-out("%=\n", i);
       repeat(next-state(collection, state));
     end unless;
   end block;
 end method;
repeat(initial-state)
  \end{verbatim}
\end{frame}

\begin{frame}[fragile]
  \frametitle{... which gets optimized to a simple loop.}
  \begin{footnotesize}
  \begin{verbatim}
while (1) {
  if ((L_state < 3)) {
    L_PCTelement = SLOT((heapptr_t)&literal_ROOT,
                        descriptor_t,
                        8 + L_state_2 * sizeof(descriptor_t));
    [...]
    L_state = L_state + 1;
  } else {
    goto block0;
  }
}
block0:;
  \end{verbatim}
  \end{footnotesize}
\end{frame}

\begin{frame}[fragile]
  \frametitle{Type unions}
  \begin{verbatim}
define constant <green-thing> =
  type-union(<frog>, <broccoli>);

define constant kermit = make(<frog>);

define method red?(x :: <green-thing>)
  #f
end;

red?(kermit) => #f
  \end{verbatim}
\end{frame}

\begin{frame}[fragile]
  \frametitle{False-or, singleton}
  \begin{verbatim}
type-union(singleton(#f, type)) == false-or(type)

define method find-foo (x) => (index :: false-or(<integer>))
 ... //returns index if found, false if not found
end method say;

type-union(symbol1, symbol2) == one-of(symbol1, symbol2)
define method say (x :: one-of(#"red", #"green", #"blue"))
  ...
end method say;
  \end{verbatim}
\end{frame}


\begin{frame}[fragile]
  \frametitle{Nonlocal exits}
  \begin{verbatim}
block (return)
  open-files();
  if (something-wrong)
    return("didn't work");
  end if;
  compute-with-files()
cleanup
  close-files();
end block
  \end{verbatim}
\end{frame}

\begin{frame}[fragile]
  \frametitle{Exceptions}
  \begin{verbatim}
block ()
  open-files();
  compute-with-files()
exception (<error>) 
  "didn't work";
cleanup
  close-files();
end block
  \end{verbatim}
\end{frame}

\begin{frame}[fragile]
  \frametitle{Library and Module}
  \begin{verbatim}
define library hello-world
  use dylan, import: all;
  use io, import: { format-out };
  export hello-world;
end library;

define module hello-world
  use dylan;
  use format-out;
  export say-hello;
end module;
  \end{verbatim}
\end{frame}

\begin{frame}
  \frametitle{Links}
  \begin{itemize}
  \item WWW: \url{http://www.gwydiondylan.org/}
  \item Dylan Programming: \url{http://www.gwydiondylan.org/books/dpg/}
  \item Dylan Reference Manual: \url{http://www.gwydiondylan.org/books/drm/}
  \item IRC: irc.freenode.net, \#dylan
  \item mailing list: gd-hackers@gwydiondylan.org
  \end{itemize}
\end{frame}

\end{document}
